%%% spelling-doc.sty
%%% Copyright 2012, 2013 Stephan Hennig
%%
%% This work may be distributed and/or modified under the conditions of
%% the LaTeX Project Public License, either version 1.3 of this license
%% or (at your option) any later version.  The latest version of this
%% license is in http://www.latex-project.org/lppl.txt
%% and version 1.3 or later is part of all distributions of LaTeX
%% version 2005/12/01 or later.
%%
%% See file README for more information.
%%
\documentclass[11pt]{article}
\usepackage{fontspec}
\defaultfontfeatures{Ligatures=TeX}
\usepackage{multicol}
\usepackage{listings}
\lstset{
  basicstyle=\ttfamily,
  columns=spaceflexible,
}
% Short-cut for non-language code snippets.
\lstMakeShortInline\|
% Short-cut for LaTeX code snippets.
\lstMakeShortInline[
language={[LaTeX]TeX},
basicstyle=\ttfamily,
]°
\usepackage{xspace}
\usepackage{array}
\usepackage{booktabs}
\usepackage[latin, UKenglish]{babel}
\usepackage[rgb, x11names]{xcolor}
\usepackage{hyperref}
\hypersetup{
  pdftitle={spelling},
  pdfauthor={Stephan Hennig},
  pdfkeywords={spell-checking, spelling, TeX, LuaTeX}
}
\hypersetup{
  english,% For \autoref.
  pdfstartview={XYZ null null null},% Zoom factor is determined by viewer.
  colorlinks,
  linkcolor=RoyalBlue3,
  urlcolor=Chocolate4,
  citecolor=DeepPink2
}
\usepackage{spelling}
\spellingreadbad{\jobname.bad}
\newcommand*{\pkg}{\textsf{spelling}}
\newcommand*{\acr}[1]{\mbox{\scshape#1}}
\newcommand*{\descr}[1]{〈\emph{#1}〉}
\newcommand*{\cmd}[1]{\mbox{\ttfamily\textbackslash#1}}
\newcommand*{\macro}[1]{\cmd{#1}\marginpar{\cmd{#1}}}
\newcommand*{\latinphrase}[1]{\foreignlanguage{latin}{\emph{#1}}}
\newcommand*{\lpcf}{\latinphrase{cf.}\xspace}
\newcommand*{\lpeg}{\latinphrase{e.\,g.}\xspace}
\newcommand*{\lpetc}{\latinphrase{etc.}\xspace}
\newcommand*{\lpie}{\latinphrase{i.\,e.}\xspace}
\begin{document}
\author{Stephan Hennig\thanks{sh2d@arcor.de}}
\title{\pkg\thanks{This document describes the \pkg\ package v0.3.}}
\maketitle


\begin{abstract}
  This package aids spell-checking of \TeX\ documents compiled with the
  Lua\TeX\ engine.  It can give visual feedback in \acr{pdf} output
  similar to \acr{wysiwyg} word processors.  The package relies on an
  external spell-checker application to check spelling of a text file
  and to output a list of bad spellings.  The package should work with
  most spell-checkers, even dumb, \TeX-unaware ones.

  \emph{Warning!  This package is in a very early state.  Everything may
    change!}
\end{abstract}

\begin{multicols}{2}
\small
% Set toc entries ragged right.  Trick taken from tocloft.pdf.
\makeatletter
\renewcommand{\@tocrmarg}{2.55em plus1fil}
\makeatother
\tableofcontents
\end{multicols}


\section{Introduction}
\label{sec:intro}

Ther%
\footnote{A fotenoot containing misspelllings.  But note how
  `misspelllings' currently slips through due to punctuation.  Some
  annoying bugs are listed in \autoref{sec:bugs}.}
%
are three main approaches to spell-checking \TeX\ documents:

\begin{enumerate}

\item checking spelling in the |.tex| source file,

\item converting a |.tex| file to another format, for which a proved
  spell-checking solution exists,

\item checking spelling after a |.tex| file has been processed by \TeX.

\end{enumerate}

All of these approaches have their strengths and weaknesses.  This
package follows the third approach, providing some unique features:

\begin{itemize}

\item In traditional solutions, text is extracted from typeset
  \acr{dvi}, \acr{ps} or \acr{pdf} files, including hyphenated words.
  Therefore, to avoid lots of false positives being reported by the
  spell-checker, hyphenation has to be switched off during the \TeX\
  run. So, one doesn't work on the original document any more.

  In contrast to that, the \pkg\ package works transparently on the
  original |.tex| source file.  Text is extracted \emph{during}
  typesetting, after Lua\TeX\ has applied its catcode and macro
  machinery, but before hyphenation takes place.

\item The \pkg\ package can highlight words with known incorrect
  spelling in \acr{pdf} output, giving visual feedback similar to
  \acr{wysiwyg} word processors.%
  \footnote{Currently, only colouring words is implemented.}

\end{itemize}


\section{Usage}
\label{sec:usage}

The \pkg\ package requires the Lua\TeX\ engine.  All functionality of
the package is implemented in Lua.  The \LaTeX\ interface, which is
described below, is effectively a wrapper around the Lua interface.

\emph{Implementing such wrappers for other formats shouldn't be too
  difficult.  The author is a \LaTeX-only user, though, and therefore
  grateful for contributions.  By the way, the \LaTeX\ package needs
  some polishing, too, \lpeg, a key-value interface is desirable.}


\subsection{Work-flow}
\label{sec:work-flow}

The work-flow of the \pkg\ package is as follows:

\begin{enumerate}

\item After the package is loaded in the preamble of a |.tex| source
  file via °\usepackage{spelling}°, a list of bad spellings is read from
  a file named \descr{jobname}|.spell.bad|, if that file exists.

\item During the Lua\TeX\ run, text is extracted from pages and all
  words are checked against the list of bad spellings.  Words with a
  known incorrect spelling are highlighted in \acr{pdf} output.

\item At the end of the Lua\TeX\ run, in addition to the \acr{pdf} file,
  a text file is written, named \descr{jobname}|.spell.txt|, by default.
  The text file should contain most of the text of the original
  document.

\item The text file is then checked by your favourite spell-checker
  application, \lpeg, Aspell or Hunspell.  The spell-checker should be
  able to write a list of bad spellings to a file.  Otherwise, visual
  feedback in \acr{pdf} output won't work.  Preferably, the file is
  named \descr{jobname}|.spell.bad|, but any other file name works just
  as well.

\item Now, there are two ways to proceed:

  \begin{enumerate}

  \item Visually minded people may just compile their document a second
    time.  This time, file \descr{jobname}|.spell.bad| is read-in again
    and the words with incorrect spelling found by the spell-checker
    should now be highlighted in \acr{pdf} output.  Checking the
    \acr{pdf} file, the necessary corrections to the |.tex| source file
    can be applied.

  \item If you're not interested in visual feedback or if your
    spell-checker doesn't provide a non-interactive mode, you can as
    well apply the necessary corrections directly to the |.tex| source
    file(s), either interactively, during the spell-checker run, or by
    looking at the final list of bad spellings in an editor (whatever
    file it was saved to).  That way, the benefit of this package is,
    that spell-checker input has already been processed by Lua\TeX, but
    contains no hyphenated words.

  \end{enumerate}

\end{enumerate}


\subsection{Word lists}
\label{sec:wordlists}

As described above, if a file \descr{jobname}|.spell.bad| exists, it is
loaded by the \pkg\ package.  The words found in the file are stored in
an internal list of bad spellings and are later used for highlighting
spelling mistakes in \acr{pdf} output.

Additionally, a second file \descr{jobname}|.spell.good| is read, if
that file exists.  The words found in that file are stored in an
internal list of good spellings.  Words in the list of good spellings
are never highlighted in \acr{pdf} output.  That is, words in the
Lua\TeX\ document are only considered spelling mistakes if they occur in
the list of bad spellings, but not in the list of good spellings.  The
list of good spellings can be used to deal with false positives (words
incorrectly reported as bad spellings by the spell-checker).

Words from additional files can be appended to the internal lists of bad
and good spellings with the \macro{spellingreadbad} and
\macro{spellingreadgood} commands.  Argument to both macros is a file
name.  File format is one word per line.  Letter case is significant.
The file must be in the \acr{utf-8} encoding.  As an example, the
command

\begin{lstlisting}[language={[LaTeX]TeX}]
\spellingreadgood{myproject.whitelist}
\end{lstlisting}
%
reads words from a file |myproject.whitelist| and adds them to the list
of good spellings.  Note, most spell-checkers provide means to deal with
unknown words via additional dictionaries.  It is recommended to
configure your spell-checker to report as few false positives as
possible.


\subsection{Highlighting spellling mistakes}
\label{sec:highlight}

\paragraph{Enabling/disabling} Highlighting spelling mistakes (words
with known incorrect spelling) in \acr{pdf} output can be toggled on and
off with command \macro{spellinghighlight}.  If the argument is |on|,
highlighting is enabled.  For other arguments, highlighting is disabled.
Highlighting is enabled, by default.

\paragraph{Colour} The colour used for highlighting bad spellings can be
determined by command \cmd{spellinghighlightcolor}.  Argument is a
colour statement in the \acr{pdf} language.  As an example, the colour
red in the \acr{rgb} colour space is represented by the statement %
|1 0 0 rg|.  In the \acr{cmyk} colour space, a reddish colour is
represented by |0 1 1 0 k|.  Default colour used for highlighting is %
|1 0 0 rg|, \lpie, red in the \acr{rgb} colour space.  \emph{Warning:
  There's currently no error checking.  Make sure, you're applying a
  valid PDF colour statement!}


\subsection{Text output}
\label{sec:textoutput}

\paragraph{Text file} After loading the \pkg\ package, at the end of the
Lua\TeX\ run, a text file is written that contains most of the document
text.  The text file is no close text rendering of the typeset document,
but serves as input for your favourite spell-checker application.  It
contains the document text in a simple format: paragraphs separated by
blank lines.  A paragraph is anything that, during typesetting, starts
with a |local_par| whatsit node in the node list representing a typeset
page of the original document, \lpeg, paragraphs in running text,
footnotes, marginal notes, (in-lined) °\parbox° commands or cells from
°p°-like table columns \lpetc

Paragraphs consist of words separated by spaces.  A word is the textual
representation of a chain of consecutive nodes of type |glyph|, |disc|
or |kern|.  Boxes are processed transparently.  That is, the \pkg\
package (highly imperfectly) tries to recognise as a single word what in
typeset output looks like a single word.  As an example, the \LaTeX\
code

\begin{center}
  \begin{tabular}{c}
\begin{lstlisting}[language={[LaTeX]TeX}]
foo\mbox{'s bar}s
\end{lstlisting}
  \end{tabular}
\end{center}
which is typeset as

\begin{center}
  foo\mbox{'s bar}s
\end{center}
is considered two words \textit{foo's} and \textit{bars}, instead of the
four words \textit{foo}, \textit{'s}, \textit{bar} and~\textit{s}.%
\footnote{This document has been compiled with a custom list of bad
  spellings, which is why the word \emph{foo's} should be highlighted.}

\paragraph{Enabling/disabling} Text output can be toggled on and off
with command \macro{spellingoutput}.  If the argument is |on|, text
output is enabled.  For other arguments, text output is disabled.  Text
output is enabled, by default.

\paragraph{File name} \hspace{0pt plus 5em} Text output file name can be
configured via command \macro{spellingoutputname}.  Argument is the new
file name.  Default text output file name is
\descr{jobname}|.spell.txt|.

\paragraph{Line length} In text output, paragraphs can either be put on
a single line or broken into lines of a fixed length.  The behaviour can
be controlled via command \macro{spellingoutputlinelength}.  Argument is
a number.  If the number is~0 or less, paragraphs are put on a single
line.  For larger arguments, the number specifies the maximum line
length.  Note, lines are broken at spaces only.  Words longer than
maximum line length are put on a single line exceeding maximum line
length.  Default line length is~72.


\subsection{Text extraction}
\label{sec:textextraction}

\paragraph{Enabling/disabling} Text extraction can be enabled and
disabled in the document via command \macro{spellingextract}.  If the
argument is |on|, text extraction is enabled.  For other arguments, text
extraction is disabled.  The command should be used in vertical mode,
\lpie, outside paragraphs.  If text extraction is disabled in the
document preamble, an empty text file is written at the end of the
Lua\TeX\ run.  Text extraction is enabled, by default.

Note, text extraction and visual feedback are orthogonal features.  That
is, if text extraction is disabled for part of a document, \lpeg, a long
table, words with a known incorrect spelling are still highlighted in
that part.


\subsection{Code point mapping}
\label{sec:cp-mapping}

As explained in \autoref{sec:textoutput}, the text file written at the
end of the Lua\TeX\ run is in the \acr{utf-8} encoding.  Unicode
contains a wealth of code points with a special meaning, such as
ligatures, alternative letters, symbols \lpetc Unfortunately, not all
spell-checker applications are smart enough to correctly interpret all
Unicode code points that may occur in a document.  For that reason, a
code point mapping feature has been implemented that allows for mapping
certain Unicode code points that may appear in a node list to arbitrary
strings in text output.  A typical example is to map ligatures to the
characters corresponding to their constituting letters.  The default
mappings applied can be found in \autoref{tab:cp-mapping}.

\begin{table}
  \centering

  \begin{tabular}{>{\ttfamily}l>{\ttfamily}l>{\ttfamily}l}
    \normalfont Unicode name & \normalfont code point & \normalfont target characters\\
    \addlinespace
    \toprule
    \addlinespace

    LATIN CAPITAL LIGATURE IJ & 0x0132 & IJ\\
    LATIN SMALL LIGATURE IJ & 0x0133 & ij\\
    LATIN CAPITAL LIGATURE OE & 0x0152 & OE\\
    LATIN SMALL LIGATURE OE & 0x0153 & oe\\
    LATIN SMALL LETTER LONG S & 0x017f & s\\
    LATIN CAPITAL LETTER SHARP S & 0x1e9e & SS\\
    LATIN SMALL LIGATURE FF & 0xfb00 & ff\\
    LATIN SMALL LIGATURE FI & 0xfb01 & fi\\
    LATIN SMALL LIGATURE FL & 0xfb02 & fl\\
    LATIN SMALL LIGATURE FFI & 0xfb03 & ffi\\
    LATIN SMALL LIGATURE FFL & 0xfb04 & ffl\\
    LATIN SMALL LIGATURE LONG S T & 0xfb05 & st\\
    LATIN SMALL LIGATURE ST & 0xfb06 & st\\
  \end{tabular}

  \caption{Default code point mappings.}
  \label{tab:cp-mapping}

\end{table}

Additional mappings can be declared by command \macro{spellingmapping}.
This command takes two arguments, a number that refers to the Unicode
code point, and a sequence of arbitrary characters that is the mapping
target.  The code point number must be in a format that can be parsed by
Lua.  The characters must be in the \acr{utf-8} encoding.

New mappings only have effect on the following document text.  The
command should therefore be used in the document preamble.  As an
example, the character |A| can be mapped to |Z| and \latinphrase{vice
  versa} with the following code:

\begin{lstlisting}[language={[LaTeX]TeX}]
\spellingmapping{65}{Z}% A => Z
\spellingmapping{90}{A}% Z => A
\end{lstlisting}

Another command \macro{spellingclearallmappings} can be used to remove
all existing code point mappings.


\subsection{Tables}
\label{sec:tables}

How do tables fit into the simple text file format that has only
paragraphs and blank lines as described in \autoref{sec:textoutput}?
What is a paragraph with regards to tables?  A whole table?  A row?  A
single cell?

By default, only text from cells in °p°(aragraph)-like columns is put on
their own paragraph, because the corresponding node list branches
contain a |local_par| whatsit node (\lpcf \autoref{sec:textoutput}).
The behaviour can be changed with the \macro{spellingtablepar} command.
This command takes as argument a number.  If the argument is~0, the
behaviour is described as above.  If the argument is~1, a blank line is
inserted before and after every table row (but at most once between
table rows).  If the argument is~2, a blank line is inserted before and
after every table cell.  By default, no blank lines are inserted.


\section{LanguageTool support}
\label{sec:languagetool}

Installing spell-checkers and dictionaries can be a difficult task if
there are no pre-built packages available for an architecture.  That's
one reason the \pkg\ package is rather spell-checker agnostic and the
manual doesn't recommend a particular spell-checker application.
Another reason is, there is no best spell-checker.  The only
recommendation the author makes is not to trust in one spell-checker,
but to use multiple spell-checkers at the same time, with different
dictionaries or, better yet, different checking engines under the hood.

Among the set of options available, LanguageTool%
\footnote{\url{http://www.languagetool.org/}}%
%
, a style and grammar checker that can also check spelling since
version~1.8, deserves some notice for its portability, ease of
installation and active development.  For these reasons, the \pkg\
package provides explicit LanguageTool support.  LanguageTool uses
Hunspell as the spell-checking engine, augmenting results with a rule
based engine and a morphological analyser (depending on the language).
The \pkg\ package can parse LanguageTool's error reports in the
\acr{xml} format, pick those errors that are spelling related and use
them to highlight bad spellings.%
\footnote{Highlighting style and grammar errors found by LanguageTool
  should be possible, but requires major restructuring of the \pkg\
  package.}


\subsection{Installation}
\label{sec:lt-installation}

Here are some brief installation instructions for the stand-alone
version of LanguageTool (tested with LanguageTool~2.0).  The stand-alone
version contains a \acr{gui} as well as a command-line interface.  For
the \pkg\ package, the latter is needed.

\begin{enumerate}

\item LanguageTool is primarily written in Java.  Make sure a recent
  Java Runtime Environment (\acr{jre}) is installed.

\item\label{enum:run-java} Open a command-line and type

\begin{lstlisting}
java -version
\end{lstlisting}
%
  If you get an error message, find out the full path to the Java
  executable (called |java.exe| on Windows) for later reference.

\item Download the stand-alone version of LanguageTool (should be a
  \acr{zip} archive).

\item Uncompress the downloaded archive to a location of your choice.

\item Open a command-line in the directory containing file
  |LanguageTool.jar| and type

\begin{lstlisting}[escapeinside=°°]
°\descr{path to}°/java -jar LanguageTool.jar --help
\end{lstlisting}
%
  Prepending the path to the Java executable is optional, depending on
  the result in step~\ref{enum:run-java}.  If you now see a list of
  LanguageTool's command-line options rush by, all is well.

\item For easier access to LanguageTool, create a small batch script and
  put that somewhere into the |PATH|.

  \begin{itemize}

  \item For users of unixoide systems, the script might look like

\begin{lstlisting}[escapeinside=°°]
#!/bin/sh
°\descr{path to}°/java -jar °\descr{path to}°/LanguageTool.jar $*
\end{lstlisting}
%
    where \texttt{\descr{path to}} should point to the Java executable
    (optional) and file |LanguageTool.jar| (mandatory).  If the script
    is named |lt.sh|, you should be able to run LanguageTool on the
    command shell by typing, \lpeg,

\begin{lstlisting}
sh lt.sh --version
\end{lstlisting}
%
    Don't forget to put the script into the |PATH|!  For other ways of
    making scripts executable, please consult the operating system
    documentation.

  \item For Windows users, the script might look like

\begin{lstlisting}[escapeinside=°°]
@echo off
°\descr{path to}°\java -jar °\descr{path to}°\LanguageTool.jar %*
\end{lstlisting}
%
    where \texttt{\descr{path to}} should point to the Java executable
    (optional) and file |LanguageTool.jar| (mandatory).  If the script
    is named |lt.bat|, you should be able to run LanguageTool on the
    command-line by typing, \lpeg,

\begin{lstlisting}
lt --version
\end{lstlisting}
%
    Don't forget to put the script into the |PATH|!

  \end{itemize}

\end{enumerate}


\subsection{Usage}
\label{sec:lt-usage}

The results of checking a text file with LanguageTool are written to an
error report, either in a human readable format or in a machine friendly
\acr{xml} format.  The \pkg\ package can only parse the latter format.
When it was said in \autoref{sec:wordlists} that the \pkg\ package reads
files \descr{jobname}|.spell.bad| and \descr{jobname}|.spell.good|, if
they exist, that was not the whole truth.  Additionally, a file
\descr{jobname}|.spell.xml| is read, if it exists.  This file should
contain a LanguageTool error report in the \acr{xml} format.  Additional
LanguageTool \acr{xml} error reports can be loaded via the
\macro{spellingreadLT} command.  Argument is a file name.  Macros
|\spellingreadLT|, |\spellingreadbad| and |\spellingreadgood| can be
used in combination in a \TeX\ file.

To check a text file and create an error report in the \acr{xml} format,
LanguageTool can be called on the command-line like this
\begin{lstlisting}[escapeinside=°°]
lt °\descr{options}° °\descr{input file}° > °\descr{error report}°
\end{lstlisting}
where \texttt{\descr{options}} is a list of options described below,
\texttt{\descr{input file}} is the text file written by the \pkg\
package in the first Lua\TeX\ run and \texttt{\descr{error report}} is
the file containing the error report.  Note, how standard output is
redirected to a file via the |>| operator.  By default, LanguageTool
writes error reports to standard output, that is, the command-line.
Redirection is a feature most operating systems provide.

\begin{itemize}

\item Option |-l| determines the language (variant) of the file to
  check.  As an example, language variant US English can be selected via
  |-l en-US|.  The full list of languages supported by LanguageTool can
  be requested via option |--list|.

\item Option |-c| determines the encoding of the input file.  Since the
  text file written by the \pkg\ package is in the \acr{utf-8} encoding,
  this part should be |-c utf-8|.

\item By default, LanguageTool outputs error reports in a human readable
  format.  The \pkg\ package can only parse error reports in the
  \acr{xml} format.  If the |--api| option is present, LanguageTool
  outputs \acr{xml} data.

\item Users that don't want to highlight bad spellings, but prefer to
  study the list of bad spellings themselves, should refer to the |-u|
  option.  But note, that with the latter option present, LanguageTool
  doesn't output pure \acr{xml} any more, even if the |--api| option is
  present.  Make sure such error reports aren't read by the \pkg\
  package.

\item If the |--help| option is present, LanguageTool shows more
  information about command-line options.

\end{itemize}

As an example, to compile a \LaTeX\ file |myletter.tex| written in
French that uses the \pkg\ package with standard settings to highlight
bad spellings and to use LanguageTool as a spell-checker, the following
commands should be typed on the command-line:

\begin{lstlisting}
lualatex myletter
lt --api -c utf-8 -l fr myletter.spell.txt > myletter.spell.xml
lualatex myletter
\end{lstlisting}


\section{Bugs}
\label{sec:bugs}

Note, this package is in a very early state.  Expect bugs!  Package
development is hosted at
\href{http://github.com/sh2d/spelling/}{\bfseries GitHub}.  The full
list of known bugs and feature requests can be found in the
\href{http://github.com/sh2d/spelling/issues/}{\bfseries issue tracker}.
New bugs should be reported there.

The most user-visible issues are listed below:

\begin{itemize}

\item There's no support for the Plain \TeX\ or Con\TeX\ formats other
  than the \acr{API} of the package's Lua modules, yet
  (\href{https://github.com/sh2d/spelling/issues/1}{issue~1}).

\item Macros provided by the \LaTeX\ package have very long names.  A
  \emph{key-value} package option interface would be much more
  user-friendly
  (\href{https://github.com/sh2d/spelling/issues/2}{issue~2}).

\item There are a couple of issues with text extraction and highlighting
  incorrect spellings:

  \begin{itemize}

  \item Text in head and foot lines is neither extracted nor highlighted
    (\href{https://github.com/sh2d/spelling/issues/7}{issue~7}).

  \item Punctuation characters are currently not stripped from words.
    For that reason, misspellings of words with leading or trailing
    punctuation will currently slip through.  This affects at least one
    word per sentence, the last one
    (\href{https://github.com/sh2d/spelling/issues/8}{issue~8}).

  \item The first word starting right after an |hlist|, \lpeg, the first
    word within an |\mbox|, is never highlighted.  It is extracted and
    written to the text file, though.  This might affect acronyms, names
    \lpetc (\href{https://github.com/sh2d/spelling/issues/6}{issue~6}).

  \item Bad spellings that are hyphenated at a page break are not
    highlighted
    (\href{https://github.com/sh2d/spelling/issues/10}{issue~10}).

  \end{itemize}


\end{itemize}

Any contributions are warmly welcome!

\bigskip
\emph{Happy \TeX ing!}


\end{document}



%%% Local Variables: 
%%% mode: latex
%%% TeX-PDF-mode: t
%%% TeX-master: t
%%% coding: utf-8
%%% End: 
