%%% spelling-doc.sty
%%% Copyright 2012 Stephan Hennig
%%
%% This work may be distributed and/or modified under the conditions of
%% the LaTeX Project Public License, either version 1.3 of this license
%% or (at your option) any later version.  The latest version of this
%% license is in http://www.latex-project.org/lppl.txt
%% and version 1.3 or later is part of all distributions of LaTeX
%% version 2005/12/01 or later.
%%
%% See file README for more information.
%%
\documentclass[11pt]{article}
\usepackage{fontspec}
\defaultfontfeatures{Ligatures=TeX}
\usepackage{multicol}
\usepackage[UKenglish]{babel}
\usepackage[rgb, x11names]{xcolor}
\usepackage{hyperref}
\hypersetup{
  pdftitle={spelling},
  pdfauthor={Stephan Hennig},
  pdfkeywords={spell-checking, spelling, TeX, LuaTeX}
}
\hypersetup{
  english,% For \autoref.
  pdfstartview={XYZ null null null},% Zoom factor is determined by viewer.
  colorlinks,
  linkcolor=RoyalBlue3,
  urlcolor=Chocolate4,
  citecolor=DeepPink2
}
\usepackage{spelling}
\newcommand*{\pkg}{\textsf{spelling}}
\newcommand*{\acr}[1]{\mbox{\scshape#1}}
\begin{document}
\author{Stephan Hennig\thanks{sh2d@arcor.de}}
\title{\pkg\thanks{This document describes the \pkg\ package v0.1.}}
\maketitle


\begin{abstract}
  This package aids spell-checking of \TeX\ documents compiled with the
  Lua\TeX\ engine.  It can give visual feedback in \acr{pdf} output
  similar to \acr{wysiwyg} word processors.  The package relies on an
  external spell-checker application to check spelling of a text file
  and to output a list of bad spellings.  The package should work with
  most spell-checkers, even dumb, \TeX-unaware ones.

  \emph{Warning!  This package is in a very early state.  Everything may
    change!}
\end{abstract}


\begin{multicols}{2}
\small
\tableofcontents
\end{multicols}


\end{document}



%%% Local Variables: 
%%% mode: latex
%%% TeX-PDF-mode: t
%%% TeX-master: t
%%% coding: utf-8
%%% End: 
